\documentclass[11pt, a4paper]{article}
\usepackage{url}
\usepackage{pgfplots}
\pgfplotsset{
	width=10cm,compat=1.9,
}
	
%\usepgfplotslibrary{external}
%\tikzexternalize
\begin{document}
\title{Programming Assignment 2:\\
	Message Passing and Cloud Computing\\
	ECSE 420: Parallel Computing\\
	Zeljko Zilic}
\author{Yarden Aran\'{e}\\
	260524831\\}
\date {October 21st, 2015}
\maketitle

\break

\section[20]{Competitor Analysis}

\textbf{A.} 
$$S(n) = \frac{1}{(1-P)+\frac{P}{n}}$$
$$\lim_{n\to\infty} S(n) = \frac{1}{1-P}$$ 
$$\frac{1}{1-P} = \frac{T_s}{T_p}$$
$$T_p = T_s * (1-P)$$
$$T_p = 300000000*(1-0.999)$$
$$T_p = 300000 s$$
$$T_p = 3.47 days$$

$$T_p = 150000000*(1-0.999)$$
$$T_p = 150000 s$$
$$T_p = 1.74 days$$

\textbf{B.}
$$S(n) = \frac{T_s}{T_p} = \frac{T_s}{T_s(1-P)+T_s(\frac{P}{n})+OV(n)}$$
$$S(n) = \frac{300000000}{300000000(1-0.999)+300000000(\frac{0.999}{n})+10n}$$
$$\frac{dS(n)}{dn} = \frac{-30000000(n^2-29970000)}{(n^2+30000n+29970000)^2}$$
$$0 = \frac{-30000000(n^2-29970000)}{(n^2+30000n+29970000)^2}$$
$$n = 5475$$
Therefore, the maximum speedup will occur when there are 5475 processors running the algorithm in parallel.
$$S(5475) = \frac{300000000}{300000000(1-0.999)+300000000(\frac{0.999}{5475})+10*5475}$$
$$S(5475) = 732.6$$
Acheiving a maximum speedup of 732.6.
$$S(5472) = \frac{T_s}{T_p}$$
$$T_p = \frac{300000000}{732.6}$$
$$T_p = 409500 s$$
$$T_p = 4.74 days$$

$$S(n) = \frac{150000000}{150000000(1-0.999)+150000000(\frac{0.999}{n})+10n}$$
$$\frac{dS(n)}{dn} = \frac{-15000000(n^2-14985000)}{(n^2+15000+14985000)^2}$$
$$0 = \frac{-15000000(n^2-14985000)}{(n^2+15000+14985000)^2}$$
$$n = 3871$$
$$S(3871) = 659.6$$
$$T_p = \frac{150000000}{659.6}$$
$$T_p = 227410 s$$
$$T_p = 2.63 days$$

\textbf{C.}

\begin{tikzpicture}
\begin{axis}[
	title={Average Speedup for cp mpi},
    axis lines = left,
    xlabel = processes,
    ylabel = speedup,
	xtick={0, 1, 2, 4},
	xmajorgrids=true,
	grid style=dashed,
]
\addplot [
	color=red,
	mark=square,
]
coordinates{
	(1, 2.5849397929)(2, 2.9449915447)(4, 3.2295739805)
};

\addlegendentry{MPI}
\end{axis}
\end{tikzpicture}

$$S(n) = \frac{1}{(1-P)+\frac{P}{n}}$$
$$S(n)(1-P)+S(n)*\frac{P}{n} = 1$$
$$P(\frac{S(n)}{n}-S(n)) = 1-S(n)$$
$$P = \frac{1-S(n)}{\frac{S(n)}{n}-S(n)}$$
$$P = \frac{\frac{1}{S(n)}-1}{\frac{1}{n}-1}$$
$$P(4) = \frac{\frac{1}{S(4)}-1}{\frac{1}{4}-1}$$
$$P(4) = \frac{\frac{1}{(3.23)}-1}{\frac{1}{4}-1}$$
$$P(4) = 0.92$$

Using Amdahl's Law with the above plotted graph, one is able to estimate the portion of the progam that is parallizable. The graph was plotted by running the program 100 times, and then calculating the speedup based on the mean.
Since the computer's architecture used to calucalte the runtimes is a 64-bit Intel® Core™ i7-3610QM CPU @ 2.30GHz × 8, it is safe to assume that maximum speedup would occur when running the algorithm with the same number of unique processors in the computer, in this case 4. Thus, the parallizable portion of the RREF algorithm is approximately 92\%. 

$$T_p = T_s * (1-P)$$
$$T_p = 99909.81*(1-0.92)$$
$$T_p = 7993 s$$
$$T_p = 2.22 hours$$

\section[20]{Deployment}
Max speedup = 659.6, half of max speedup = 329.9, three-quarters of max speedup = 494.7.

\begin{tabular}{| l | l | l | l|}
	\hline
	\textbf{speedup} & \textbf{processes} & \textbf{time (hours)} & \textbf{deployment cost}\\ \hline
	659.6 & 3871 & 63.2 & \$2446.5\\ \hline
	494.7 & 1050 & 84.2 & \$884.1\\ \hline
	329.9 & 500 & 126.3 & \$631.5\\ \hline
\end{tabular}



\end{document}
